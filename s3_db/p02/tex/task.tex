\section{Мета}

\begin{enumerate}
	\item Створення бази даних шляхом визначення схеми БД
	\item Навчитися проектувати бази даних, вводити і редагувати структуру таблиць та дані в таблицях
	\item Вивчити DDL-команди SQL для роботи з таблицями (створення, модифікації та видалення таблиць)
	\item Вивчити використовувані в SQL засоби для підтримки цілісності даних та їх практичне застосування
	\item Вивчити основні принципи керування обліковими записами та ролями
\end{enumerate}

\section{Робота рієлторської компанії}

У рієлторську компанію звертаються клієнти, які бажають продати/купити чи зняти/здати нерухомість
у оренду. Компанія визначає рієлтора, який буде вести справи клієнтів, на підставі поточної
завантаженості працівників. Виділений рієлтор реєструє клієнта, його контактні дані, тип та
адресу нерухомості, вартість та інші характеристики, а також статус (здача в оренду, продаж,
здача в оренду або продаж). Склад і кількість характеристик може змінюватись відповідно до типу
нерухомості. При виникненні запиту на нерухомість рієлтор зв'язується з клієнтами і погоджує
зручний час і дату огляду нерухомості.

У разі згоди потенційного орендаря на оренду чи покупця на покупку нерухомості рієлтор зв'язується
з ним і погоджує дату оформлення договору про оренду чи покупку. Для здійснення угоди рієлтор оформляє
необхідні дозволи, документи, контракти та договори, після чого передає їх у центральний апарат
компанії для кінцевого нотаріального засвідчення. Вартість послуг рієлтора складає або половину
місячної вартості оренди нерухомості у випадку здачі в оренду, або 2\% від суми угоди при продажі нерухомості.

Клієнти можуть публікувати власні оголошення на оренду чи продаж, але вони публікуються тільки після
проходження модерації адміністратором. В кінці кожного місяця формується звіт про надані рієлторські
послуги та загальний прибуток по кожному типу нерухомості. За результатами аналізу звіту центральний
апарат компанії приймає рішення щодо розширення або звуження штату працівників. Пошуком пропозицій
на здачу в оренду чи продаж нерухомості також займається центральний апарат фірми.

\section{Постановка задачі}

\begin{enumerate}
	\item Розробити SQL-скрипти для:

	      \begin{enumerate}
		      \item створення БД згідно з розробленою в роботі №1 ER-моделлю;
		      \item створення таблиць в БД засобами мови SQL. Передбачити наявність обмежень для підтримки
		            цілісності та коректності даних, котрі зберігаються та вводяться;
		      \item встановлення зв’язків між таблицями засобами мови SQL;
		      \item зміни в структурах таблиць, обмежень засобами мови SQL (до 10 різних за суттю запитів
		            для декількох таблиць (використати DDL-команди SQL));
		      \item видалення окремих елементів таблиць/обмежень або самих таблиць засобами мови SQL
		            (до 10 різних за суттю запитів (використати DDL-команди SQL));
		      \item визначити декілька (2-3) типів користувачів, котрі будуть працювати з розробленою базою
		            даних. Для кожного користувача визначити набір привілеїв, котрі він буде мати;
		      \item для визначених типів користувачів створити відповідні ролі та наділити їх необхідними привілеями;
		      \item створити по одному користувачу в базі даних для кожного типу та присвоїти їм відповідні ролі.
	      \end{enumerate}

	\item Згенерувати схему даних засобами СУБД
	\item Імпортувати дані в створену БД з використанням засобів СУБД, а не DML SQL
\end{enumerate}
