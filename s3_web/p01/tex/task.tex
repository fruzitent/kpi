\section{Мета}

Ознайомлення з мовою розмітки гіпертексту HTML та базовими засобами CSS
для візуалізації веб-сторінок.

\section{Завдання}

\begin{enumerate}
      \item Оберіть із 16 варіантів власний варіант виконання роботи відповідно до вашого порядкового номера у списку групи.
      \item Засобами HTML створіть веб-сторінку заданого вигляду згідно з вашим варіантом, використовуючи CSS для позиціювання та візуалізації окремих елементів веб-сторінки.
      \item Позначки «X» та «Y» замініть вигаданими назвами чи гаслами з використанням тегів заголовку.
      \item Серед блоків з позначками «1», «2», «3», «4», «5», «6», «7» оберіть один для подальшого розміщення меню. В решті блоків позначки замініть кількома абзацами довільного тексту, проконтролюйте дотримання заданого вигляду.
      \item Створіть умовний веб-сайт за допомогою розмноження отриманої вебсторінки до 5 екземплярів та розміщення в цих веб-сторінках меню з посиланнями на ці веб-сторінки.
      \item Використайте теги UL, OL, LI, IMG, A, MAP для додавання на окремі вебсторінки списків, зображень, посилань та карт посилань із них.
      \item Переконайтеся в збереженні горизонтальних пропорцій та відсутності помилок відображення блоків у веб-сторінках на різних обсягах змістового наповнення та на різних розмірах вікна переглядача.
      \item Продемонструйте функціонування новоствореного набору веб-сторінок в якості веб-сайта через локальну файлову систему.
      \item Скористайтесь GitHub Pages або обраним вами хостінгом для публікації вашого веб-сайта в інтернеті.
      \item Порівняйте способи доступу до локального сайта через локальну файлову систему і до веб-сайта на хостінгу через інтернет. При потребі підкоригуйте роботу посилань.
      \item Продемонструйте функціонування веб-сайта в інтернеті.
\end{enumerate}
